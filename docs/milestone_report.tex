% Milestone Report (Draft)
% Uses NeurIPS conference template from docs/Styles (as per instructions/rubric in Canvas).
% Main-text page limit: \textbf{max 5 pages} (references do not count). Appendix allowed \textbf{figures only} -> (as per instructions/rubric in Canvas).
\documentclass{article}

% --- Packages ---
\usepackage[utf8]{inputenc}
\usepackage[T1]{fontenc}
\usepackage{lmodern}
\usepackage{amsmath,amssymb}
\usepackage{graphicx}
\usepackage{booktabs}
\usepackage{siunitx}
\usepackage{hyperref}
% Load NeurIPS style from local Styles directory (relative to this file).
% Choose one of: final | preprint | anonymous; keep nonatbib to use manual bibliography here.
\usepackage[final,nonatbib]{Styles/neurips_2025}
\usepackage{xcolor}

	\title{CS 7643 Milestone Report: Learning Costmap Generation from RGB\,\&\,Depth for Mobile Robot Navigation}
\author{Rut Santana \and Ibrahim Alshayeb \and Vineet Kulkarni \and Meera Ranjan}
\date{November 2nd, 2025}

\begin{document}
\maketitle

\begin{abstract}
We aim to learn traversability costmaps directly from RGB\,+\,Depth images and assess them with both perception metrics and planner-level outcomes. Since the proposal, we implemented the end-to-end data pipeline (NYU and KITTI), dataset loaders, training/evaluation scripts, losses/metrics, and configuration files. This draft summarizes progress, clarifies hypotheses, outlines the methodology, and provides placeholders for preliminary results to be filled by teammates.
\end{abstract}

\section{Introduction}
\textbf{Problem \/ Motivation.} Safe autonomous navigation requires reliable costmaps marking free space and obstacles for planners (e.g., A*, RRT*). Classical pipelines generated from depth/LiDAR can be brittle across domains and sensors. We study learning costmaps from RGB\,+\,Depth (RGBD) to improve robustness while preserving planner compatibility.

\textbf{Inputs\/Outputs \/ Training Conditions.} Input is a single RGBD image $I\in\mathbb{R}^{H\times W\times 4}$; output is a local, egocentric costmap $C\in[0,1]^{64\times 64}$. We also evaluate a binarized occupancy map by thresholding $C$. Supervision uses costmaps derived from depth-based heuristics. Optimization uses Adam with a composite objective of L1 and Dice losses (and an optional boundary-aware term).

\textbf{Datasets and splits (paired .npz).} NYU: 523 train, 131 val. KITTI: 438 train, 433 val. All pairs are standardized to 4-channel inputs and $64\times64$ targets.

	\textbf{Goal.} Compare classical depth-to-cost mapping with UNet, ViT, and Hybrid CNN+Transformer models, measuring MAE, IoU, Precision\/Recall\/F1 on held-out splits and, later, planner performance. \\
	\textbf{Planner status:} Planner-in-the-loop evaluation with A* and RRT* is \emph{not yet performed in this milestone} and is planned as future work.

\section{Related Work}
Our proposal emphasized works that learn traversability or BEV maps from vision while connecting predictions to planning utility. We keep foundational dense-prediction references (U-Net; ViT) for context and highlight the proposal-cited papers below. We will format final citations per NeurIPS (Bib\TeX recommended).
\begin{itemize}
  \item \textbf{TerrainNet} (Meng et al., 2023): fuses semantic and geometric cues for high-speed off-road traversability; stresses boundary fidelity and planning-aware metrics.
  \item \textbf{U-Net adaptation for Mars rovers} (Qiu \& Lloyd, 2025): simplified U-Net variant highlighting efficiency for resource-limited platforms.
  \item \textbf{Trans4Map} (Chen et al., 2022): ViT-based egocentric-to-BEV mapping improving global consistency.
  \item \textbf{Camera-only BEV pipelines} (Bochare, 2025; Chang et al., 2024): show promise without LiDAR but exhibit sensitivity to depth errors and calibration.
  \item \textbf{Preference-conditioned costmaps} (Mao et al., 2025): condition costmaps on user preferences; complementary to our fixed traversability objective.
  \item \textbf{Foundational background}: U-Net for dense prediction (Ronneberger et al., 2015) and ViT (Dosovitskiy et al., 2021).
\end{itemize}
Compared to these, our focus is a unified pipeline that (i) standardizes supervision from depth-derived costmaps on indoor/outdoor datasets, (ii) compares UNet, ViT, and Hybrid encoders under identical training and metrics, and (iii) evaluates both perception quality and planner-level outcomes (A*, RRT*).

\section{Methodology (Tentative Technical Approach)}
\subsection{Data Processing and Labels}
We implemented scripts to prepare NYU and KITTI, discover RGB\/Depth pairs, and generate targets. Depth is projected\/processed into an egocentric grid and converted to a normalized costmap $C\in[0,1]^{64\times64}$. The same heuristic is used across datasets to ensure label consistency.

\subsection{Models}
We compare three families with a common lightweight decoder to a 1-channel output:
\begin{itemize}
  \item \textbf{UNet (CNN baseline)}: encoder--decoder with skip connections.
  \item \textbf{ViT encoder}: patch embedding + transformer encoder + convolutional upsampling decoder.
  \item \textbf{Hybrid}: CNN stem, transformer bottleneck, CNN decoder for local\,+\,global fusion.
\end{itemize}
All models produce a $64\times64$ map (resize applied if needed).

\subsection{Objective and Metrics}
Training loss uses $\mathcal{L} = \lambda_{\ell_1}\,\mathcal{L}_{\ell_1} + \lambda_{d}\,\mathcal{L}_{\text{Dice}} (+ \lambda_b\,\mathcal{L}_{\text{boundary}})$.
Evaluation reports Mean Absolute Error (MAE) on continuous cost and IoU, Precision, Recall, and F1 on a binarized map (threshold $\tau{=}0.5$ by default).

\subsection{Evaluation Protocol}
We train per-dataset (NYU, KITTI) and evaluate on held-out validation splits. Optional cross-domain tests (NYU$\rightarrow$KITTI and vice versa) assess generalization. For this milestone, we report \emph{perception metrics only}; planner-in-the-loop evaluation (A*, RRT*) of predicted costmaps is future work.

\subsection{Hypotheses}
\begin{description}
  \item[H1 (Modality)] RGBD inputs outperform RGB-only for IoU and MAE, holding model and schedule fixed.
  \item[H2 (Architecture)] The Hybrid model achieves higher IoU than pure UNet or ViT at similar parameter budgets.
  \item[H3 (Objective)] L1+Dice improves IoU over L1-only by emphasizing occupied regions and boundaries.
\end{description}

\section{Baseline Results \& Trials of Your Method (Placeholders)}
This section is designed for the milestone's required baseline and preliminary results. Replace placeholders with actual numbers, figures, and brief analyses.

\subsection{Baselines}
\begin{itemize}
  \item \textbf{Classical depth-to-cost (no learning)}: thresholding\/morphology\/distance transform to produce costmap.
  \item \textbf{Sanity checks}: all-free and all-obstacle predicted maps to validate metric behavior.
\end{itemize}

\subsection{Quantitative Results (to be filled)}
% Example table layout; duplicate for KITTI and NYU as needed.
\begin{table}[h]
  \centering
  \caption{NYU validation metrics (placeholder). Threshold $\tau{=}0.5$.}
  \label{tab:nyu_results}
  \begin{tabular}{lcccccc}
    \toprule
    Method & MAE $\downarrow$ & IoU $\uparrow$ & Precision & Recall & F1 & Params (M) \\
    \midrule
    Classical depth-to-cost & -- & -- & -- & -- & -- & -- \\
    UNet & -- & -- & -- & -- & -- & -- \\
    ViT & -- & -- & -- & -- & -- & -- \\
    Hybrid & -- & -- & -- & -- & -- & -- \\
    \bottomrule
  \end{tabular}
\end{table}

% Duplicate table for KITTI to satisfy common reporting across datasets
\begin{table}[h]
  \centering
  \caption{KITTI validation metrics (placeholder). Threshold $\tau{=}0.5$.}
  \label{tab:kitti_results}
  \begin{tabular}{lcccccc}
    	oprule
    Method & MAE $\downarrow$ & IoU $\uparrow$ & Precision & Recall & F1 & Params (M) \\
    \midrule
    Classical depth-to-cost & -- & -- & -- & -- & -- & -- \\
    UNet & -- & -- & -- & -- & -- & -- \\
    ViT & -- & -- & -- & -- & -- & -- \\
    Hybrid & -- & -- & -- & -- & -- & -- \\
    \bottomrule
  \end{tabular}
\end{table}

\subsection{Implementation Notes (to be filled)}
% ~One paragraph or subsection per teammate; include design choices and observations.
\subsubsection*{UNet Implementation}
Design summary, training settings, challenges, and key results.

\subsubsection*{ViT Implementation}
Design summary, training settings, challenges, and key results.

\subsubsection*{Hybrid Implementation}
Design summary, training settings, challenges, and key results.

\subsection{Qualitative Results}
Include side-by-side panels: RGB, Depth, baseline costmap, predicted costmap, and binarized occupancy overlay.

\section{Next Steps (Placeholder: to be properly filled)}
\textbf{Near-term (1--2 weeks).} Implement classical baseline; complete UNet; run first comparisons on NYU and KITTI; perform a light hyperparameter search (lr, batch size, $\lambda_d$).

\textbf{Mid-term (3--4 weeks).} Implement ViT and Hybrid; cross-domain tests; ablations for H1--H3 (modality, architecture, objective). Optional planner smoke tests using A* and RRT* with predicted costmaps.

\textbf{Endgame.} Error analysis on boundary regions and thin obstacles; compute FLOPs\/latency; finalize figures and writing.

\section*{Reproducibility}
All code and configuration files are in the repository (data processing scripts; PyTorch training and evaluation; YAML configs). Processed pair counts: NYU 523/131 (train/val); KITTI 438/433. Data sharing is via three archives (\texttt{kitti\_raw.zip}, \texttt{nyu\_raw.zip}, \texttt{processed.zip}) with a fetch script for teammates.

\section*{References (placeholder)}
\begin{thebibliography}{9}\itemsep0pt
% Foundational
\bibitem{unet} Ronneberger, O., Fischer, P., Brox, T. U-Net: Convolutional Networks for Biomedical Image Segmentation. MICCAI, 2015.
\bibitem{vit} Dosovitskiy, A., et al. An Image is Worth 16x16 Words: Transformers for Image Recognition at Scale. ICLR, 2021.
% Proposal-cited works
\bibitem{meng2023} Meng, X., et al. TerrainNet: Visual Modeling of Complex Terrain for High-speed, Off-road Navigation. 2023.
\bibitem{qiu2025} Qiu, R., and Lloyd, V. Reduced Image Classes in Modified U-Net for Mars Rover Navigation. 2025.
\bibitem{chen2022} Chen, C., et al. Trans4Map: Revisiting Holistic BEV Mapping from Egocentric Images with Vision Transformers. 2022.
\bibitem{bochare2025} Bochare, A. Camera-Only Bird’s Eye View Perception: A Neural Approach to LiDAR-Free Mapping for Autonomous Vehicles. 2025.
\bibitem{chang2024} Chang, et al. BEVMap: Map-Aware BEV Modeling for 3D Perception. 2024.
\bibitem{mao2025} Mao, L., et al. PACER: Preference-conditioned All-terrain CostMap Generation. 2025.
% Additional background
\bibitem{monodepth2} Godard, C., Aodha, O. M., Brostow, G. Monodepth2. ICCV, 2019.
\bibitem{deepdriving} Chen, C., et al. Deep Driving. ICCV, 2015.
\bibitem{vin} Tamar, A., et al. Value Iteration Networks. NeurIPS, 2016.
\end{thebibliography}

% Appendix allowed \textbf{only for figures}; no additional text per guidelines/instructions/rubric.
\appendix
\section*{Appendix (Figures Only)}
% Example multi-panel qualitative figure placeholder (remove if not used).
\begin{figure}[h]
  \centering
  % \includegraphics[width=0.95\linewidth]{figs/qualitative_panel_example}
  \caption{Qualitative examples (placeholder): RGB, Depth, classical baseline costmap, predicted costmap, binarized occupancy overlay.}
\end{figure}

\end{document}
