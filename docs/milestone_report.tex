% Milestone Report (Draft)
% Uses NeurIPS conference template from docs/Styles (as per instructions/rubric in Canvas).
% Main-text page limit: \textbf{max 5 pages} (references do not count). Appendix allowed \textbf{figures only} -> (as per instructions/rubric in Canvas).
\documentclass{article}

% --- Packages ---
\usepackage[utf8]{inputenc}
\usepackage[T1]{fontenc}
\usepackage{lmodern}
\usepackage{amsmath,amssymb}
\usepackage{graphicx}
\usepackage{booktabs}
\usepackage{siunitx}
\usepackage{hyperref}
% Load NeurIPS style from local Styles directory (relative to this file).
% Choose one of: final | preprint | anonymous; keep nonatbib to use manual bibliography here.
\usepackage[final,nonatbib]{Styles/neurips_2025}
\usepackage{xcolor}

	\title{CS 7643 Milestone Report: Learning Costmap Generation from RGB\,\&\,Depth for Mobile Robot Navigation}
\author{Rut Santana \and Ibrahim Alshayeb \and Vineet Kulkarni \and Meera Ranjan}
\date{November 2nd, 2025}

\begin{document}
\maketitle

\begin{abstract}
We aim to learn traversability costmaps directly from RGB\,+\,Depth images and assess them with both perception metrics and planner-level outcomes. Since the proposal, we implemented the end-to-end data pipeline (NYU and KITTI), dataset loaders, training/evaluation scripts, losses/metrics, and configuration files, and we report preliminary NYU/KITTI results for UNet, ViT, and a Hybrid CNN+Transformer. The report clarifies our hypotheses, details the methodology and evaluation protocol, and outlines the plan toward planner-in-the-loop assessment.
\end{abstract}

\section{Introduction}
\textbf{Problem \/ Motivation.} Safe autonomous navigation requires reliable costmaps marking free space and obstacles for planners (e.g., A*, RRT*). Classical pipelines generated from depth/LiDAR can be brittle across domains and sensors. We study learning costmaps from RGB\,+\,Depth (RGBD) to improve robustness while preserving planner compatibility.

\textbf{Inputs\/Outputs \/ Training Conditions.} Input is a single RGBD image $I\in\mathbb{R}^{H\times W\times 4}$; output is a local, egocentric costmap $C\in[0,1]^{64\times 64}$. We also evaluate a binarized occupancy map by thresholding $C$. Supervision uses costmaps derived from depth-based heuristics. Optimization uses Adam with a composite objective of L1 and Dice losses (and an optional boundary-aware term).

\textbf{Datasets and splits (paired .npz).} NYU: 523 train, 131 val. KITTI: 438 train, 433 val. All pairs are standardized to 4-channel inputs and $64\times64$ targets.

	\textbf{Goal.} Compare classical depth-to-cost mapping with UNet, ViT, and Hybrid CNN+Transformer models, measuring MAE, IoU, Precision\/Recall\/F1 on held-out splits and, later, planner performance. \\
	\textbf{Planner status:} Planner-in-the-loop evaluation with A* and RRT* is \emph{not yet performed in this milestone} and is planned as future work.

\section{Related Work}
Our proposal emphasized learning traversability or BEV maps from vision and connecting predictions to planning utility. \textit{TerrainNet} \cite{meng2023} fuses semantic and geometric cues for high-speed off-road traversability and stresses boundary fidelity and planning-aware metrics. A simplified U-Net for Mars rovers \cite{qiu2025} highlights efficiency for resource-limited platforms. Transformer-based BEV mapping such as Trans4Map \cite{chen2022} improves global consistency; camera-only pipelines \cite{bochare2025,chang2024} show promise without LiDAR but can be sensitive to depth errors and calibration. Preference-conditioned costmaps \cite{mao2025} offer flexibility that is complementary to our fixed traversability objective. We ground our models in foundational dense prediction with U-Net \cite{unet} and ViT \cite{vit}. Compared to prior work, our focus is a unified pipeline that standardizes supervision from depth-derived costmaps on indoor/outdoor datasets, compares UNet, ViT, and Hybrid encoders under identical training and metrics, and evaluates perception quality with a planned extension to planner-level outcomes (A*, RRT*). Consistent with our proposal, we emphasize controlled \emph{internal} comparisons, and we do not report numeric cross-paper comparisons because labels, splits, and protocols differ across works.

\section{Methodology (Tentative Technical Approach)}
\subsection{Data Processing and Labels}
We implemented scripts to prepare NYU and KITTI, discover RGB\/Depth pairs, and generate targets. Depth is projected\/processed into an egocentric grid and converted to a normalized costmap $C\in[0,1]^{64\times64}$. The same heuristic is used across datasets to ensure label consistency.

\subsection{Models}
We compare three families with a common lightweight decoder to a 1-channel output. The \textbf{UNet} baseline uses an encoder--decoder with skip connections. The \textbf{ViT} variant employs patch embedding, a transformer encoder, and a convolutional upsampling decoder. The \textbf{Hybrid} model combines a CNN stem with a transformer bottleneck followed by a CNN decoder to fuse local and global context. All models produce a $64\times64$ map (resize applied if needed).

\subsection{Objective and Metrics}
Training loss uses $\mathcal{L} = \lambda_{\ell_1}\,\mathcal{L}_{\ell_1} + \lambda_{d}\,\mathcal{L}_{\text{Dice}} (+ \lambda_b\,\mathcal{L}_{\text{boundary}})$.
Evaluation reports Mean Absolute Error (MAE) on continuous cost and IoU, Precision, Recall, and F1 on a binarized map (threshold $\tau{=}0.5$ by default).

\subsection{Evaluation Protocol}
We train per-dataset (NYU, KITTI) and evaluate on held-out validation splits. Optional cross-domain tests (NYU$\rightarrow$KITTI and vice versa) assess generalization. For this milestone, we report \emph{perception metrics only}; planner-in-the-loop evaluation (A*, RRT*) of predicted costmaps is future work.

\subsection{Hypotheses}
We test three hypotheses. \textbf{H1 (Modality):} RGBD inputs outperform RGB-only for IoU and MAE when holding architecture and schedule fixed. \textbf{H2 (Architecture):} the Hybrid model achieves higher IoU than pure UNet or ViT at similar parameter budgets. \textbf{H3 (Objective):} combining L1 with Dice improves IoU over L1-only by emphasizing occupied regions and boundaries.

\section{Preliminary Results}
This section satisfies the milestone's requirement to report baseline and preliminary results. Following our proposal, we focus on controlled \emph{internal} comparisons among UNet, ViT, and Hybrid under identical data, losses (L1+Dice), and metrics (MAE on continuous cost; IoU/Precision/Recall/F1 at threshold $\tau{=}0.5$). Planner-level outcomes (success, collisions, path cost, planning time) will be evaluated in the next phase using A*/RRT* on predicted costmaps.

\subsection{Baselines}
We include a classical depth-to-cost baseline (thresholding, morphology, distance transform) and simple sanity checks (all-free and all-obstacle predictions) to contextualize metric behavior.

\subsection{Quantitative Results}
% NYU results table
\begin{table}[h]
  \centering
  \caption{NYU validation metrics. Threshold $\tau{=}0.5$.}
  \label{tab:nyu_results}
  \begin{tabular}{lcccccc}
    \toprule
    Method & MAE $\downarrow$ & IoU $\uparrow$ & Precision & Recall & F1 & Params (M) \\
    \midrule
    UNet & 0.0139 & 0.9168 & 0.9526 & 0.9607 & 0.9566 & 4.2 \\
    ViT & 0.0063 & 0.9750 & 0.9880 & 0.9860 & 0.9870 & -- \\
    Hybrid & 0.0088 & 0.9680 & 0.9830 & 0.9840 & 0.9840 & -- \\
    \bottomrule
  \end{tabular}
\end{table}

% KITTI results table
\begin{table}[h]
  \centering
  \caption{KITTI validation metrics. Threshold $\tau{=}0.5$.}
  \label{tab:kitti_results}
  \begin{tabular}{lcccccc}
    	oprule
    Method & MAE $\downarrow$ & IoU $\uparrow$ & Precision & Recall & F1 & Params (M) \\
    \midrule
    UNet (KITTI only) & 0.2083 & 0.4803 & 0.6148 & 0.6989 & 0.6384 & 4.2 \\
    UNet (NYU $\rightarrow$ KITTI TL) & 0.2015 & 0.4999 & 0.6313 & 0.6956 & \textbf{0.6514} & 4.2 \\
    ViT & 0.1890 & 0.4940 & 0.5990 & \textbf{0.7230} & 0.6500 & -- \\
    Hybrid & \textbf{0.1740} & \textbf{0.5270} & \textbf{0.7090} & 0.5850 & 0.6360 & -- \\
    \bottomrule
  \end{tabular}
\end{table}

\paragraph{Summary.} On NYU, \emph{ViT} achieves the best overall performance across MAE/IoU/F1, with \emph{Hybrid} close behind and \emph{UNet} trailing by ~5--6 IoU points. On KITTI, \emph{Hybrid} attains the best MAE (0.174) and IoU (0.527) with strong precision, while recall is lower; \emph{UNet} with transfer learning from NYU yields the best F1 (0.651) and improves over the KITTI-only UNet by +1.96 IoU points and +1.3 F1 points; \emph{ViT} reaches the highest recall (0.723) and competitive F1 (0.650). The trends suggest indoor scenes (dense depth, regular geometry) favor global context from transformers, while outdoor KITTI benefits from Hybrid's local+global fusion and from cross-domain pretraining.

\subsection{Implementation Notes}
	\textbf{UNet.} Architecture based on \href{https://github.com/milesial/Pytorch-UNet}{milesial/Pytorch-UNet} with light edits. Training used Adam, L1+Dice loss, and a final \texttt{sigmoid} applied prior to L1/Dice and thresholding for metrics ($\tau=0.5$). A transfer-learning run (pretrain on NYU, fine-tune on KITTI) improved KITTI metrics: IoU +1.96 pts (0.4803 $\rightarrow$ 0.4999), F1 +1.3 pts (0.6384 $\rightarrow$ 0.6514), and MAE from 0.2083 $\rightarrow$ 0.2015. Profiled model: \textbf{4.2M} params, \textbf{6.72 GFLOPs}, \textbf{1.75 ms/frame} on Tesla V100 (\textasciitilde571 FPS).\par
	\textbf{ViT.} Patch embedding with transformer encoder and a convolutional upsampling decoder to a 1-channel output. Same optimizer and loss. ViT excelled on NYU (MAE 0.0063, IoU 0.975, F1 0.987) and achieved the highest recall on KITTI (0.723), indicating stronger sensitivity to thin/fragmented obstacles but with some over-segmentation (lower precision).\par
	\textbf{Hybrid.} CNN stem with a transformer bottleneck and CNN decoder for local/global fusion. Heads match UNet/ViT for a fair comparison. Hybrid is competitive on NYU and leads MAE/IoU on KITTI (MAE 0.174, IoU 0.527) with high precision (0.709). Lower recall than ViT suggests potential benefit from boundary or focal terms and augmentation targeted at outdoor clutter.

\subsection{Qualitative Results}
Include side-by-side panels: RGB, Depth, baseline costmap, predicted costmap, and binarized occupancy overlay.

\subsection{Problem Decomposition and Experimental Plan}
Our three hypotheses translate into concrete experiments. For \textbf{H1 (Modality)}, we will hold the architecture and training schedule fixed and compare RGBD versus RGB-only inputs on both NYU and KITTI, reporting MAE, IoU, precision/recall, and F1 at $\tau{=}0.5$ alongside PR curves to analyze calibration across thresholds. For \textbf{H2 (Architecture)}, we will match parameter budgets across UNet, ViT, and Hybrid by adjusting base channel widths and transformer depth, then compare accuracy, FLOPs, and latency; this isolates representational differences from capacity. For \textbf{H3 (Objective)}, we will ablate L1 versus L1+Dice and optionally add a boundary-aware term; we expect L1+Dice to improve occupied-region fidelity and thin obstacles. Training follows the same splits used here (NYU: 523/131; KITTI: 438/433). Risks include domain shift and class imbalance outdoors; we mitigate with transfer learning (NYU$\rightarrow$KITTI) and augmentation (random brightness/contrast, flips, small perspective jitter). Success criteria are: (i) NYU F1 $\geq$ 0.98 for the best model; (ii) KITTI IoU $\geq$ 0.53 with F1 $\geq$ 0.65; and (iii) real-time feasibility on a single GPU (\(\leq\) 10 ms/frame for 256$\times$256 RGBD).

\subsection{Classical Baseline and Training Setup}
The classical depth-to-cost baseline thresholds depth to mark obstacles, applies morphology to close holes and dilate by a robot footprint, and computes a distance transform to produce a continuous cost map normalized to $[0,1]$. This provides an interpretable, planner-compatible reference and informs error modes (such as thin obstacles, sensor noise, etc.). All learning models use Adam, L1+Dice loss, and \texttt{sigmoid} activation before regression and thresholding; metrics default to $\tau{=}0.5$ with PR curves computed for sensitivity analysis. Unless otherwise noted, we train with batch sizes tuned per GPU memory, cosine or step LR schedules around 1e-3 initial rate, and early stopping on validation F1. We profile params, FLOPs, and latency to quantify deployability; UNet runs at 1.75 ms/frame on a V100 at $\sim$571 FPS, while ViT/Hybrid trade a modest latency increase for improved accuracy.

\begin{figure}[h]
  \centering
  % Teaser placeholder (no external file needed): three framed boxes
  \fbox{\rule{0pt}{1.1in}\rule{0.28\linewidth}{0pt}}\hfill
  \fbox{\rule{0pt}{1.1in}\rule{0.28\linewidth}{0pt}}\hfill
  \fbox{\rule{0pt}{1.1in}\rule{0.28\linewidth}{0pt}}\\[0.3ex]
  \small Left: RGB; middle: depth; right: predicted costmap (Hybrid). A draft teaser illustrating inputs and outputs. Full qualitative panels will include baseline and binarized overlays.
  \caption{Draft teaser figure illustrating the input modalities and an example predicted costmap.}
  \label{fig:teaser}
\end{figure}

\section{Next Steps}
  \textbf{Near-term (1--2 weeks).} Run a classical depth-to-cost baseline and add it to Tables~\ref{tab:nyu_results}--\ref{tab:kitti_results}; perform a hyperparameter sweep (lr, batch size, $\lambda_d$, data augmentations); and enable boundary or focal terms to raise KITTI recall for Hybrid without sacrificing precision.\par
  \textbf{Mid-term (3--4 weeks).} Conduct ablations for H1--H3: modality (RGB vs RGBD), architecture (UNet vs ViT vs Hybrid at matched parameters), and objective (L1 vs L1+Dice vs +Boundary); run cross-domain tests (NYU$\rightarrow$KITTI and KITTI$\rightarrow$NYU) including transfer learning; and add calibration analysis (precision--recall across thresholds).\par
  \textbf{Endgame.} Perform planner-in-the-loop evaluation (A* on grids, RRT* in continuous space) using predicted costmaps; report success, collisions, path cost, and planning time; finalize error analysis (thin obstacles, distant clutter), profile models (params, FLOPs, latency), and complete writing.

\section*{Reproducibility}
Code and configuration files are in the repository. Training and evaluation live in \texttt{src/train/train.py}; datasets in \texttt{src/data/dataset\_npz.py}; losses and metrics are configured via YAML in \texttt{configs/}. Example configs include \texttt{train\_nyu\_unet.yaml}, \texttt{train\_nyu\_hybrid.yaml}, \texttt{train\_kitti\_unet.yaml}, and \texttt{train\_kitti\_hybrid.yaml}. Processed pair counts are NYU 523/131 and KITTI 438/433. We evaluate MAE on continuous costs and IoU/precision/recall/F1 at $\tau{=}0.5$ by default, with threshold sweeps for PR curves.

\begin{thebibliography}{9}\itemsep0pt
% Foundational
\bibitem{unet} Ronneberger, O., Fischer, P., Brox, T. U-Net: Convolutional Networks for Biomedical Image Segmentation. MICCAI, 2015.
\bibitem{vit} Dosovitskiy, A., et al. An Image is Worth 16x16 Words: Transformers for Image Recognition at Scale. ICLR, 2021.
% Proposal-cited works
\bibitem{meng2023} Meng, X., et al. TerrainNet: Visual Modeling of Complex Terrain for High-speed, Off-road Navigation. 2023.
\bibitem{qiu2025} Qiu, R., and Lloyd, V. Reduced Image Classes in Modified U-Net for Mars Rover Navigation. 2025.
\bibitem{chen2022} Chen, C., et al. Trans4Map: Revisiting Holistic BEV Mapping from Egocentric Images with Vision Transformers. 2022.
\bibitem{bochare2025} Bochare, A. Camera-Only Bird’s Eye View Perception: A Neural Approach to LiDAR-Free Mapping for Autonomous Vehicles. 2025.
\bibitem{chang2024} Chang, et al. BEVMap: Map-Aware BEV Modeling for 3D Perception. 2024.
\bibitem{mao2025} Mao, L., et al. PACER: Preference-conditioned All-terrain CostMap Generation. 2025.
% Additional background
\bibitem{monodepth2} Godard, C., Aodha, O. M., Brostow, G. Monodepth2. ICCV, 2019.
\bibitem{deepdriving} Chen, C., et al. Deep Driving. ICCV, 2015.
\bibitem{vin} Tamar, A., et al. Value Iteration Networks. NeurIPS, 2016.
\end{thebibliography}

% Appendix allowed \textbf{only for figures}; no additional text per guidelines/instructions/rubric.
\appendix
\section*{Appendix (Figures Only)}
% Example multi-panel qualitative figure placeholder (remove if not used).
\begin{figure}[h]
  \centering
  % \includegraphics[width=0.95\linewidth]{figs/qualitative_panel_example}
  \caption{Qualitative examples (placeholder): RGB, Depth, classical baseline costmap, predicted costmap, binarized occupancy overlay.}
\end{figure}

\end{document}
